% Options for packages loaded elsewhere
\PassOptionsToPackage{unicode}{hyperref}
\PassOptionsToPackage{hyphens}{url}
%
\documentclass[
  11pt,
]{article}
\usepackage{lmodern}
\usepackage{amssymb,amsmath}
\usepackage{ifxetex,ifluatex}
\ifnum 0\ifxetex 1\fi\ifluatex 1\fi=0 % if pdftex
  \usepackage[T1]{fontenc}
  \usepackage[utf8]{inputenc}
  \usepackage{textcomp} % provide euro and other symbols
\else % if luatex or xetex
  \usepackage{unicode-math}
  \defaultfontfeatures{Scale=MatchLowercase}
  \defaultfontfeatures[\rmfamily]{Ligatures=TeX,Scale=1}
\fi
% Use upquote if available, for straight quotes in verbatim environments
\IfFileExists{upquote.sty}{\usepackage{upquote}}{}
\IfFileExists{microtype.sty}{% use microtype if available
  \usepackage[]{microtype}
  \UseMicrotypeSet[protrusion]{basicmath} % disable protrusion for tt fonts
}{}
\makeatletter
\@ifundefined{KOMAClassName}{% if non-KOMA class
  \IfFileExists{parskip.sty}{%
    \usepackage{parskip}
  }{% else
    \setlength{\parindent}{0pt}
    \setlength{\parskip}{6pt plus 2pt minus 1pt}}
}{% if KOMA class
  \KOMAoptions{parskip=half}}
\makeatother
\usepackage{xcolor}
\IfFileExists{xurl.sty}{\usepackage{xurl}}{} % add URL line breaks if available
\IfFileExists{bookmark.sty}{\usepackage{bookmark}}{\usepackage{hyperref}}
\hypersetup{
  pdftitle={The role of present-day speciation in modern dynamics of vertebrate diversity},
  hidelinks,
  pdfcreator={LaTeX via pandoc}}
\urlstyle{same} % disable monospaced font for URLs
\usepackage[margin=1in]{geometry}
\usepackage{graphicx,grffile}
\makeatletter
\def\maxwidth{\ifdim\Gin@nat@width>\linewidth\linewidth\else\Gin@nat@width\fi}
\def\maxheight{\ifdim\Gin@nat@height>\textheight\textheight\else\Gin@nat@height\fi}
\makeatother
% Scale images if necessary, so that they will not overflow the page
% margins by default, and it is still possible to overwrite the defaults
% using explicit options in \includegraphics[width, height, ...]{}
\setkeys{Gin}{width=\maxwidth,height=\maxheight,keepaspectratio}
% Set default figure placement to htbp
\makeatletter
\def\fps@figure{htbp}
\makeatother
\setlength{\emergencystretch}{3em} % prevent overfull lines
\providecommand{\tightlist}{%
  \setlength{\itemsep}{0pt}\setlength{\parskip}{0pt}}
\setcounter{secnumdepth}{-\maxdimen} % remove section numbering
\usepackage{setspace}\doublespacing
\usepackage{lineno}\linenumbers
\usepackage[utf8]{inputenc}
\usepackage[T1]{fontenc}
\usepackage{biblatex}
\usepackage{booktabs}
\usepackage{longtable}
\usepackage{array}
\usepackage{multirow}
\usepackage{wrapfig}
\usepackage{float}
\usepackage{colortbl}
\usepackage{pdflscape}
\usepackage{tabu}
\usepackage{threeparttable}
\usepackage{threeparttablex}
\usepackage[normalem]{ulem}
\usepackage{makecell}
\usepackage{xcolor}

\title{The role of present-day speciation in modern dynamics of vertebrate
diversity}
\author{}
\date{\vspace{-2.5em}}

\begin{document}
\maketitle

\emph{Alternative title}: Present-day speciation rates do not generate
modern terrestrial vertebrate cradles and museums\\
\emph{Alternative title}: Present-day speciation rates do not generate
modern dynamics of terrestrial vertebrate diversity\\
\emph{Alternative title}: Recent tetrapod cradles are not the result of
present-day speciation

\begin{center}
\textbf{H{\'{e}}ctor Tejero-Cicu{\'{e}}ndez$^{1,*}$,  Iris Men{\'{e}}ndez$^{2}$, and Jiří Šmíd$^{3}$}
\end{center}

\begin{center}02 febrero, 2023\end{center}

\(^{1}\)Evolution and Conservation Biology research Group, Department of
Biodiversity, Ecology and Evolution. Faculty of Biology. Universidad
Complutense de Madrid, 28040, Madrid, Spain

\(^{2}\)Museum für Naturkunde, Leibniz Institute for Evolution and
Biodiversity Science, Berlin, Germany

\(^{3}\)Charles University, Prague, Czech Republic

\(^{*}\)Correspondence: Héctor Tejero-Cicuéndez
\href{mailto:cicuendez93@gmail.com}{\nolinkurl{cicuendez93@gmail.com}}

\newpage

\hypertarget{abstract}{%
\section{Abstract}\label{abstract}}

Evolutionary and ecological dynamics differ across regions of Earth and
across clades of tree of life.

\newpage

\hypertarget{introduction}{%
\section{1. Introduction}\label{introduction}}

The evolutionary and ecological processes underlying global patterns of
biodiversity have always been a central subject of study for
evolutionary biologists. Much of today's discussion about the geography
of biodiversity dynamics stems from theoretical and conceptual
developments on the characterization of regional conditions leading to
different ways of diversity assembly. The terms `cradle' and `museum' to
refer to regions of high instability, heterogeneity and species turnover
(cradles) and regions of long-lasting environmental stability and
taxonomic diversity (museums) have been very popular in the macroecology
literature since Stebbins {[}1{]} proposed this metaphor. Even though
the dichotomous interpretation of these terms has resulted in an
inappropriate simplification or directly in a wrong use {[}2{]}, it is
still important to identify the geographically uneven distribution of
diversity dynamics to search for their ultimate historical, ecological,
and evolutionary drivers. \hfill\break

The modern use of the terms museum and cradle and their original meaning
{[}2{]}. \hfill\break

Whether these particular words are used or not, there are regions that
show clearly distinct diversity patterns in terms of number of species
(species richness) and how closely related those species are
(phylogenetic diversity). Although the geographic patterns of species
richness and phylogenetic diversity are ever more well-characterized,
the role of different factors in generating such patterns is not clear
in most cases. Diversification rates are one of the most frequently
invoked factors when characterizing biodiversity patterns, but their
effect remains ambiguous in the context of global patterns of vertebrate
diversity. \hfill\break

In this study, BLABLABLA\ldots{}

\hypertarget{materials-and-methods}{%
\section{2. Materials and Methods}\label{materials-and-methods}}

\hypertarget{results}{%
\section{3. Results}\label{results}}

\hypertarget{discussion}{%
\section{4. Discussion}\label{discussion}}

The fact that the places with high speciation rates at present are not
the places with more closely related diversity relative to their species
richness might indicate that historical diversification trends are
driving geographical patterns of global diversity.

\newpage

\hypertarget{references}{%
\section*{References}\label{references}}
\addcontentsline{toc}{section}{References}

\setlength{\parindent}{-0.25in} \setlength{\leftskip}{0.25in}
\setlength{\parskip}{8pt} \noindent

\hypertarget{refs}{}
\leavevmode\hypertarget{ref-Stebbins1974}{}%
1. Stebbins GL. 1974 \emph{Flowering plants: Evolution above the species
level}. Harvard University Press. See
\url{https://books.google.cz/books?id=LMbWngEACAAJ}.

\leavevmode\hypertarget{ref-Vasconcelos2022}{}%
2. Vasconcelos T, O'Meara BC, Beaulieu JM. 2022 Retiring "Cradles" and
"Museums" of Biodiversity. \emph{The American Naturalist} \textbf{199}.
(doi:\href{https://doi.org/10.32942/osf.io/sxah8}{10.32942/osf.io/sxah8})

\newpage

\hfill\break

\textbf{Acknowledgments}:

\textbf{Funding Statement}: This work was funded in part by BLABLABLA to
JŠ. IM was funded by the Alexander von Humboldt Foundation through a
Humboldt Research Fellowship. HT-C is supported by a ``Juan de la Cierva
- Formación'' postdoctoral fellowship (FJC2021-046832-I) funded by
MCIN/AEI/10.13039/501100011033 and by the European Union
NextGenerationEU/PRTR.

\textbf{Data availability statement}: All the data used in this study
are available elsewhere and cited accordingly throughout the manuscript.
The scripts for implementing all analyses and generating the figures in
this manuscript can be found in the Supplementary Material and in a
GitHub repository (and on DRYAD upon acceptance).

\textbf{Competing interests}: The authors declare no competing
interests.

\newpage

\hypertarget{figures}{%
\section{Figures}\label{figures}}

Figure 1. Conceptual representation of the expected relationship between
speciation rates and phylogenetic diversity (PD) relative to species
richness.

Figure 2. Relationship between DR rates and the residuals resulting from
a regression of \texttt{PD\ \textasciitilde{}\ Richness} for all four
groups of terrestrial vertebrates.

Figure 3. Geographic distribution of cradles and museums for terrestrial
vertebrates. Silhouettes extracted from `phylopic' (www.phylopic.org).

\newpage

\begin{figure}

{\centering \includegraphics[width=1\linewidth]{figures/Fig1} 

}

\caption{Conceptual representation of the expected relationship between speciation rates and phylogenetic diversity (PD) relative to species richness.}\label{fig:unnamed-chunk-1}
\end{figure}

\newpage

\begin{figure}

{\centering \includegraphics[width=1\linewidth]{figures/Fig2} 

}

\caption{Relationship between DR rates and the residuals resulting from a regression of `PD ~ Richness` for all four groups of terrestrial vertebrates. }\label{fig:unnamed-chunk-2}
\end{figure}

\newpage

\begin{figure}

{\centering \includegraphics[width=1\linewidth]{figures/Fig3} 

}

\caption{Geographic distribution of cradles and museums for terrestrial vertebrates. Silhouettes extracted from 'phylopic' (www.phylopic.org).}\label{fig:unnamed-chunk-3}
\end{figure}

\newpage

\end{document}
